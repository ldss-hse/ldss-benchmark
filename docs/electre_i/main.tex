\documentclass{article}

% Language setting
% Replace `english' with e.g. `spanish' to change the document language
\usepackage[russian]{babel}

% Set page size and margins
% Replace `letterpaper' with `a4paper' for UK/EU standard size
\usepackage[letterpaper,top=2cm,bottom=2cm,left=3cm,right=3cm,marginparwidth=1.75cm]{geometry}

% Useful packages
\usepackage{amsmath}
\usepackage{graphicx}
\usepackage[colorlinks=true, allcolors=blue]{hyperref}

\title{Краткое описание основных шагов метода ELECTRE I}
\author{Александр Владимирович Демидовский}

\begin{document}
\maketitle

\begin{abstract}
Данный документ содержит конспект основных шагов метода многокритериального анализа решений  ELECTRE I. Даный документ носит ознакомительный характер и выполнен в рамках подготовки диссертации на соискание степени кандидата компьютерных наук.
\end{abstract}

\section{Введение}

    Методы ELECTRE (ELimination and Choice Expressing REality) были изначально предложены
    Benayoun R. в \cite{scm:electre:2:benayoun1966manual} и затем значительно доработаны
    исследователем Roy B. \cite{scm:electre:3:roy1968classement}. В общем, данные подходы
    позволяют определять доминирование альтернативных решений друг относительно друга через
    призму анализа их согласованности. Рассмотрим процесс принятия решения с момента, когда
    начинается сбор оценок экспертов по каждой альтернативе по каждому критерию. В результате
    формируется матрица A, которая имеет следующую форму:

    \begin{equation}
    \label{eq:decision_matrix}
        A = \begin{bmatrix}
            x_{11}&  x_{12}& \hdotsfor{4} & x_{1n}\\
            x_{21}&  x_{22}& \hdotsfor{4} & x_{2n}\\
        \hdotsfor{7}\\
        x_{m1}&  x_{m2}& . & . & . & . & x_{mn}\\
        \end{bmatrix}
    \end{equation}

    где \(x_{ij}\) обозначает оценку, данную по \(i\)-той альтернативе и по \(j\)-тому критерию.

    Оригинальный метод состоит из 9 последовательных шагов
    \cite{scm:electre:1:hwang1981methods}. Каждый из них описан ниже.

\section{Описание рассматриваемого метода: ELECTRE I}
    \subsection{Расчет нормализованной матрицы решений}
        Во время данного шага нормализация происходит по столбцу в силу того, что столбец
        соответствует заданному критерию и ему соответствует одна единица измерения.

        \begin{equation}
            \label{eq:norm_decision_matrix}
            r_{ij} = \frac{x_{ij}}{\sqrt{\sum_{i=1}^{m}x_{ij}^2}}
        \end{equation}

        В результате получаем нормализованную матрицу решений \(R\).

        \[
            R = \begin{bmatrix}
             r_{11}&  r_{12}& \hdotsfor{4} & r_{1n}\\
             r_{21}&  r_{22}& \hdotsfor{4} & r_{2n}\\
            \hdotsfor{7}\\
            r_{m1}&  r_{m2}& . & . & . & . & r_{mn}\\
            \end{bmatrix}
        \]

    \subsection{Расчет взвешенной нормализованной матрицы решений}
        Веса \(w=(w_1,w_2,\dots,w_n)\) назначаются каждому критерию так, что требуется умножить
        каждую \(i\)-тый столбец нормализованной матрицы решения на \(j\)-тый вес.
        Для сохранения матричной формы расчетов, представим веса как диагональную матрицу \(W\):

        \[
            W = \begin{bmatrix}
                w_{1}&  0 & \hdotsfor{2} & 0\\
                0 &  w_{2}& \hdotsfor{2} & 0\\
                \hdotsfor{5}\\
                0 &  0 & . & . & w_{n}\\
            \end{bmatrix}
        \]

        Затем, взвешенная нормализованная матрица решений получается в результате матричного
        умножения нормализованной матрицы решений и диагональной матрицы весов:

        \begin{equation}
        \label{eq:weighted_norm_decision_matrix}
            \begin{alignedat}{2}
                V &= \begin{bmatrix}
                    r_{11}&  r_{12}& \hdotsfor{4} & r_{1n}\\
                    r_{21}&  r_{22}& \hdotsfor{4} & r_{2n}\\
                    \hdotsfor{7}\\
                    r_{m1}&  r_{m2}& . & . & . & . & r_{mn}\\
                \end{bmatrix} * \begin{bmatrix}
                    w_{1}&  0 & \hdotsfor{2} & 0\\
                    0 &  w_{2}& \hdotsfor{2} & 0\\
                    \hdotsfor{5}\\
                    0 &  0 & . & . & w_{n}\\
                \end{bmatrix} \\
                &= \begin{bmatrix}
                    r_{11}*w_{1}&  r_{12} * w_{2}& \hdotsfor{4} & r_{1n}*w_{n}\\
                    r_{21}*w_{1}&  r_{22} * w_{2}& \hdotsfor{4} & r_{2n}*w_{n}\\
                    \hdotsfor{7}\\
                    r_{m1}*w_{1}&  r_{m2} * w_{2}& . & . & . & . & r_{mn}*w_{n}\\
                \end{bmatrix}
                \end{alignedat}
        \end{equation}

    \subsection{Построение множеств согласия и несогласия}
        Этот шаг является одним из ключевых в данном методе. Для каждой пары альтернатив
        \(A_k\) и \(A_l\) происходит разделение множества критериев \(J\) на два подмножества:
        первое (\(C_{kl}\)) содержит те критерии, по которым альтернатива \(A_k\) является
        предпочтительной относительно альтернативы \(A_l\), а второе подмножество (\(D_{kl}\))
        является дополнением первого до полного множества альтернатив:

        \begin{equation}
            \begin{split}
            C_{kl} &= \{j \mid x_{kj} \geq  x_{lj}\} \\
            D_{kl} &= \{j \mid x_{kj} <  x_{lj}\} = J - C_{kl}
            \end{split}
        \end{equation}

    \subsection{Расчет матрицы согласия}
        Как только множества согласия определены, происходит расчет индекса согласия между
        двумя парами альтернатив \(A_k\) и \(A_l\). Это сумма весов входящих в это
        множество критериев:

        \begin{equation}
            c_{kl} = \frac{\sum_{j \epsilon C_{kl}}w_{j}}{\sum_{j}^{n}w_{j}}
        \end{equation}

        В результате получается матрица \(C\):
        \[
            \begin{bmatrix}
                -&  c_{12}& \hdotsfor{4} & c_{1n}\\
                c_{21}&   - & \hdotsfor{4} & c_{2n}\\
                \hdotsfor{7}\\
                c_{m1}&  c_{m2}& . & . & . & . & -\\
            \end{bmatrix}
        \]

    \subsection{Расчет матрицы несогласия}
        Как только множества несогласия определены, происходит расчет индекса несогласия между
        парами альтернатив \(A_k\) и \(A_l\). Общая мотивация заключается в понимании того,
        насколько первый критерий хуже, чем второй:

        \begin{equation}
            d_{kl} = \frac
            {\max_{j \epsilon D_{kl}} \mid v_{kj} - v_{lj} \mid}
            {\max_{j \epsilon J} \mid v_{kj} - v_{lj} \mid}
        \end{equation}

        В результате получается матрица \(D\):
        \[
            \begin{bmatrix}
                -&  d_{12}& \hdotsfor{4} & d_{1n}\\
                d_{21}&   - & \hdotsfor{4} & d_{2n}\\
                \hdotsfor{7}\\
                d_{m1}&  d_{m2}& . & . & . & . & -\\
            \end{bmatrix}
        \]

    \subsection{Построение матрицы доминирования согласия}
        Для того, чтобы можно было начать отбрасывать не доминирующие альтернативы, происходит
        преобразование матрицы согласия \(C\) в двоичную матрицу \(F\) с применением особого
        порогового значения (средний индекс согласия) \(\widetilde{c}\):

        \begin{equation}
            \widetilde{c} = \sum_{k=1, k \neq l}^{m}\sum_{l=1, l \neq k}^{m}
            \frac{c_{kl}}{m*(m-1)}
        \end{equation}

        Матрица доминирования согласия \(F\) строится согласно правилу:

        \begin{equation}
            f_{kl} = \begin{cases}
                1,& \text{если } c_{kl}\geq \widetilde{c}\\
                0,              & \text{иначе}
            \end{cases}
        \end{equation}

    \subsection{Построение матрицы доминирования несогласия}
        Данный шаг происходит аналогично предыдущему шагу. В результате получается матрица
        доминирования несогласия \(G\).

    \subsection{Построение общей матрицы доминирования}
        Во время данного шага происходит слияние матриц согласия и несогласия через поэлементное
        умножение этих матриц. Результирующая матрица \(E\) и есть общая матрица доминирования.

        \begin{equation}
            E = \left \| e_{kl} \right \| = \left \| f_{kl} * g_{kl}\right \|
        \end{equation}


    \subsection{Удаление наименее привлекательных альтернатив}
        Данный шаг требует анализа того, какие альтернативы оказываются доминирующими, и в
        результате остаются одна или несколько альтернатив, предлагаемых ЛПР как наилучшие.

\bibliographystyle{alpha}
\bibliography{sample}

\end{document}
